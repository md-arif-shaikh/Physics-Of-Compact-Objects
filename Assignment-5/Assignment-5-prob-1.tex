\documentclass[]{revtex4-2}
\usepackage{amsmath}

\begin{document}
    \title{Assignment 5: Physics of Compact Objects}
    \author{Md Arif Shaikh}
    \affiliation{International Centre for Theoretical Sciences, Bengaluru}
    \email{arifshaikh.astro@gmail.com}
    \date{\today}
    \maketitle
    \section*{Problem 1}
    Show that, in the white dwarf interior, the Coulomb energy per free electron is
    \begin{equation}
      \label{eq:coulomb-energy}
      \epsilon_C = - \frac{9}{10} \left(\frac{4\pi}{3}\right)^{1/3}Z^{2/3}e^2 n_e^{1/3},
    \end{equation}
    where $Z$ is the atomic number, $e$ is the elementary charge and $n_e$ is the number density of free electrons.

\vspace{0.5cm}
    
\noindent {\bfseries Answer:} As temperature decreases, $T\to0$, the ions are located in a lattice that maximizes the inter-ion separation. We can consider a spherical shell of the lattice of volume $4\pi r_0^3/3 = 1/n_N$, where $n_N$ is the number density of the nuclei. The total energy of any one sphere is the sum of potential energies due to electron-electron ($e-e$) interactions and electron-ion ($e-i$) interactions.

To assemble a uniform sphere of $Z$ electrons requires energy
\begin{equation}
  \label{eq:e-e}
  E_{e-e} = \int_0^{r_0} \frac{q dq}{r} = \int_0^{r_0} - \frac{Ze (r^3/r_0^3)d ( - Ze (r^3/r_0^3))}{r} = Z^2e^2 \frac{3}{r_0^6}\int_0^{r_0} r^4 dr = Z^2e^2 \frac{3}{r_0^6}\frac{r_0^5}{5} = \frac{3}{5}\frac{Z^2e^2}{r_0}.
\end{equation}
On the other hand, to assemble the electron sphere about the central nucleus of charge $Ze$ requires energy
\begin{equation}
  \label{eq:e-i}
  E_{e-i} = Ze \int_0^{r_0} \frac{dq}{r} = - Ze \int_0^{r_0} \frac{3 dr Z e r}{r_0^3} = - \frac{3}{2}\frac{Z^2e^2}{r_0}.
\end{equation}
So, the total Coulomb energy of the shell is
\begin{equation}
  \label{eq:E_tot}
  E_C = E_{e-e} + E_{e-i} =  \frac{3}{5}\frac{Z^2e^2}{r_0} - \frac{3}{2}\frac{Z^2e^2}{r_0} = - \frac{9}{10}\frac{Z^2e^2}{r_0}.
\end{equation}
So the Coulomb energy per electron is
\begin{equation}
  \label{eq:epsilon_c}
  \epsilon_C = \frac{E_C}{Z} = -\frac{9}{10}\frac{Z e^2}{r_0} = - \frac{9}{10}\frac{Z e^2}{(3 Z/4\pi n_e)^{1/3}} = - \frac{9}{10} \left(\frac{4\pi}{3}\right)^{1/3}Z^{2/3}e^2n_e^{1/3},
\end{equation}
where we have used
\begin{equation}
  \label{eq:n_e}
  \frac{4\pi}{3} r_0^3 n_e = Z. 
\end{equation}
\end{document}